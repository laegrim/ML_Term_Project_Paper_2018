\section{Introduction}
	Data characterizes the modern world. Corresponding to exponential growth in computing power over time, and the related proliferation of sensors and interactions with computing systems, the volume of recorded data has exploded. It is estimated that in 2010 there was more than 1 Zettabyte of recorded unstructured data, and that as of 2018 there exists more than 2 Zettabytes structured data and 20 Zettabytes structured data \cite{dataage2025}\cite{dataRizzatti}. While much of this deluge is comprised of highly specific and narrowly specialized data, not useful for making decisions at a personal level, there still is a wealth of data - far more than is human interpretable - which could be used to inform such decisions. Recommendation systems are predictive tools designed specifically for this use case, helping pare from a glut of options a select few choices based on available data.\par
The Netflix Prize competition \cite{netflixPrize}, started in 2006 and finalized in 2009, is, along with the publishing of the ImageNet dataset in 2009 \cite{imagenet}, credited as one of the events that encouraged the perception of a revolution in the field of machine learning both among professional audiences and the public at large \cite{imagenetReasons}. In the near decade since team BellKor's Pragmatic Chaos won the Netflix Prize competition \cite{netflixPrize}, rapid advances, especially in Deep Learning, have continued to push boundaries in machine learning and recomendation systems. The Netflix dataset remains an important benchmark dataset for recomendation systems.\par
In this paper we will discuss recent improvements in the field of recomendation systems and attempt to benchmark some of those improvments on the Netflix dataset. Section 2 will cover some important backround material on the field of recomendation systems. Section 3 will discuss some recent and related results. Section 4 will talk about our experimental setup, while Section 5 will include the specific experiments carried out. Section 6 discusses our results, conclusions, and will cover our expectations for future work.

