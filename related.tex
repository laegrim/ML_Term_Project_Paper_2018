\section{Related}
The winning team of the Netflix Prize competition, team BellKor's Pragmatic Chaos, used a large ensemble of smaller models, including decision trees, support vector machines, restricted boltzmann machines, and regression models in their solution, but did not use any deep learning models - deep learning models were not yet popular or practical \cite{Piotte 09th Pragmatic}\cite{bigChaos}\cite{Koren091the}. Since then, deep learning recommendation systems have been the subject of extensive research. \cite{DBLP:journals/corr/ZhangYS17aa} is a survey paper detailing several such deep learning recommendation systems, including models employing deep feedforward neural networks (MLP), recurrent neural networks (RNN), convolutional neural networks (CNN), deep auto-encoders (AE), generative adversarial networks (GAN), and composite models. Given project time restrictions, our we narrowed our focus to MLP and RNN based deep learning recommendation systems. 
\subsection{Takeaway From Related Work}
There were four main themes throughout the body of related work:
\begin{itemize}
\item Meaningful object representation
\item Explicit hierarchical abstraction
\item Control of data flow
\item Use of domain or expert knowledge
\end{itemize}
Black box solutions - such solving the problem purely through application of large, overparameterized, neural networks - do not appear in the literature on deep learning recommendation systems in the same way that they seem to in the fields of image segmentation or recognition. Deep learning recommendation systems seem to require to be tailored for the exact context and medium of the interaction. As such methods from implicit and explicit interaction have somewhat limited interoperability. It seems the trend in recommendations is moving towards implicit interactions which, given our dataset limits some of what we can do. Given time constraints, we chose to focus our efforts on meaningful object representation.  